\documentclass{article}

\author{Adrian Vollmer}
\title{Cosmomathica documentation}
\date{\today}

\begin{document}
\maketitle

\abstract{Cosmomathica is a package for Mathematica by Wolfram Research,
Inc., providing a set of tools for computations in cosmology. It offers
access to the 3rd party software packages CAMB, Halofit+, Cosmic Emulator
and a transfer function fitting formula via a MathLink interface.}

% \section{Features}
% 
% The central part of cosmomathica is the definition of a cosmological model.
% The expression \verb#Cosmology# consists of a list of rules that define the
% parameters of the model that is used. These rules represent the default
% options of all relevant cosmological functions. At the beginning of a
% calculation, you would set the default options to your fiducial model, and
% if necessary you can evaluate any function at some other point in the
% parameter space by specifying the corresponding options.
% 
% Some functions (such as the comoving distance) can be somewhat
% computationally intensive, since it involves integration or solving a
% differential equation. These functions are being computed only once in form
% of an interpolated function which is then evaluated when called after the
% first use.

\section{External packages}

It is possible to only use cosmomathica as an interface to the 3rd party
packages CAMB\cite{Lewis}, Halofit+\cite{Smith}, Cosmic Emulator\cite{we}
and a transfer function fitting formula\cite{EisensteinHu} via a MathLink.
Since CAMB has been written in Fortran, a wrapper module has been developed.
The functions that call the MathLink have purposefully been kept as simple
as possible with a minimum number of parameters, returning only the raw data
from the respective code without adjustment of units or anything similar.

% As part of cosmomathica, these interface functions can also be called in a
% unified way with consistent units, parameters, and so on.

On top of that, the Halofit algorithm has been reimplemented in Mathematica
to apply the non-linear corrections to arbitrary power spectra.

\section{List of functions}

\verb#CAMB[OmegaC, OmegaB, OmegaL, h, w]#

\verb#Transfer[omegaM, fBaryon, Tcmb, h]#

\verb#Halofit[OmegaM, OmegaL, gammaShape, sigma8, ns, betaP, z0]#

\verb#CosmicEmu[omegaM, omegaB, sigma8, ns, w]#

\section{Examples}

% SetCosmology[\{\}]

% Plot[Table[ComovingDistance[z,OmegaK->k],{k,{-1,0,1}}],{z,0,3}]

\section{Credit}



\end{document}
